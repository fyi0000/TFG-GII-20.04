\capitulo{2}{Objetivos del proyecto}

\section{Objetivos generales}

\begin{itemize}
	\item Profundizar en la aplicación del \emph{Deep Learning} y redes neuronales en el mundo industrial.
	\item Asegurar una ejecución eficiente y con un rendimiento adecuado.
	\item Análisis de los resultados, procesado y presentación sencilla de los mismos.
	\item Hacer posible la ejecución de la herramienta final en múltiples entornos.
\end{itemize}

\section{Objetivos técnicos}

\begin{itemize}
    \item Introducción y aprendizaje de la librería de \emph{Facebook} \emph{Detectron 2} para la detección y segmentación de imágenes.
    \item Analizar la fase de entrenamiento y optimizar el modelo resultante.
    \item Desarrollar una aplicación web en \emph{Flask} que permita el uso sencillo del modelo
    \item Facilitar la accesibilidad de la web con la ayuda de \emph{Docker} para hacer posible su ejecución en múltiples entornos.
    \item Introducirse al uso práctico de \emph{Github} y la organización del proyecto.
\end{itemize}

\section{Objetivos personales}

\begin{itemize}
    \item Profundizar en el mundo del \emph{Machine Learning} y conocer hasta qué punto está presente en el día a día.
    \item Aumentar los conocimientos ya adquiridos de \emph{Python} con el uso de más librerías.
    \item Adentrarse en el mundo del desarrollo web introduciéndose tanto en \emph{Frontend}, de cara al usuario y la presentación del diferente contenido web, como en el \emph{Backend}, como funciona la parte del servidor y el tratamiento de la información facilitada por el usuario.
    \item Analizar las necesidades de un trabajador del sector para intentar aportar el mayor número de facilidades con la herramienta.
\end{itemize}
