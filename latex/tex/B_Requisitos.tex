\apendice{Especificación de Requisitos}

\section{Introducción}
A continuación se exponen los objetivos generales del proyecto y del producto final así como los diferentes requisitos y casos de uso.

\section{Objetivos generales}

\begin{itemize}
    \item Obtención de un modelo que se comporte correctamente con las imágenes propias
    \item Integración en una aplicación web que permita la correcta ejecución del modelo de forma sencilla para el usuario
    \item Mostrar los resultados de una forma amigable y permitir la conservación de los mismos
\end{itemize}

\section{Catalogo de requisitos}
Se detallan ahora los requisitos funcionales de la aplicación.

\begin{itemize}
    \item \textbf{RF 1} Permitir la ejecución del modelo sobre una radiografía mostrando las detecciones.
    \begin{itemize}
	    \item \textbf{RF 1.1} Poder subir la imagen objetivo del usuario a la aplicación.
	    \item \textbf{RF 1.2} Mostrar previsualización de la imagen.
	    \item \textbf{RF 1.3} Muestra de los resultados de forma interactiva.
	    \item \textbf{RF 1.4} Personalizar el valor límite o confianza de los defectos.
	\end{itemize}

	\item \textbf{RF 2} Permitir la descarga de resultados.
    \begin{itemize}
	    \item \textbf{RF 2.1} Descargar la máscara binaria generada.
	    \item \textbf{RF 2.2} Descargar el resultado interactivo de la figura \emph{Plotly}.
	    \item \textbf{RF 2.3} Descargar los resultados en forma de composición.
	\end{itemize}
	
	\item \textbf{RF 3} Mostrar información de las detecciones.
    \begin{itemize}
	    \item \textbf{RF 3.1} Clasificar los defectos por tamaño y etiquetarlos en la figura.
	    \item \textbf{RF 3.2} Mostrar el área del defecto.
	\end{itemize}
	
	\item \textbf{RF 4} Generar un histórico de detecciones.
    \begin{itemize}
	    \item \textbf{RF 4.1} Registrar en un fichero las detecciones por fecha y tamaño de defectos.
	    \item \textbf{RF 4.2} Generar un gráfico de áreas a partir del histórico.
	\end{itemize}

\end{itemize}

\clearpage

\section{Especificación de requisitos}

\subsection{Diagrama de casos de uso}

\begin{figure}[htb]
	\centering
	\includegraphics[width=1.0\textwidth]{casosdeUso}
	\caption[Diagrama de casos de uso]{Diagrama de casos de uso}
\end{figure}

\tablaSmallSinColores{C0: Cargar imagen a la aplicación}{p{3cm} p{.75cm} p{9.5cm}}{casodeUso0}{
  \multicolumn{3}{l}{C0: Cargar imagen a la aplicación.} \\
 }
 {
  Descripción                       &  \multicolumn{2}{p{10.25cm}}{Permite cargar una imagen a la aplicación para su uso.} \\\hline
  \multirow{2}{3.5cm}{Requisitos}   &   \multicolumn{2}{p{10.25cm}}{RF 1.1} \\\cline{2-3}
                                    &   \multicolumn{2}{p{10.25cm}}{RF 1.2}\\\hline
  Precondiciones                    &   \multicolumn{2}{p{10.25cm}}{Aplicación inicializada}   \\\hline
  \multirow{2}{3.5cm}{Secuencia normal}  & Paso & Acción \\\cline{2-3}
                                         & 1    & Se selecciona una imagen.
  \\\cline{2-3}
                                         & 2    & Se pulsa el botón subir y se muestra la profesionalización.
  \\\cline{2-3}
                                         & 3    & La imagen se guarda en el directorio de la aplicación.
                                         \\\hline
  Postcondiciones                        & \multicolumn{2}{p{10.25cm}}{La imagen se guarda en el directorio correspondiente y se muestra la previsualización.} \\\hline
  Excepciones                        & \multicolumn{2}{p{10.25cm}}{No se ha seleccionado ningún fichero. El formato de fichero no es ni .png, .jpg o .jpeg}\\
}

\clearpage
  
\tablaSmallSinColores{C1: Ejecutar detección}{p{3cm} p{.75cm} p{9.5cm}}{casodeUso1}{
  \multicolumn{3}{l}{C1: Ejecutar detección.} \\
 }
 {
  Descripción                       &  \multicolumn{2}{p{10.25cm}}{Se ejecuta el modelo sobre la imagen y se muestran los resultados.} \\\hline
  \multirow{2}{3.5cm}{Requisitos}   &   \multicolumn{2}{p{10.25cm}}{RF 1} \\\cline{2-3}
                                    &   \multicolumn{2}{p{10.25cm}}{RF 1.4}\\\hline
                                    
  Precondiciones                    &   \multicolumn{2}{p{10.25cm}}{Imagen correctamente cargada en la aplicación.}   \\\hline
  \multirow{2}{3.5cm}{Secuencia normal}  & Paso & Acción \\\cline{2-3}
                                         & 1    & Se selecciona en el \emph{slider} la confianza deseada para la detección.
  \\\cline{2-3}
                                         & 2    & Se pulsa el botón Aceptar y se procede a la detección.
\\\cline{2-3}
                                         & 3    & Tras la carga, se muestran los resultados en forma de gráfico \emph{Plotly}.
                                         \\\hline
  Postcondiciones                        & \multicolumn{2}{p{10.25cm}}{Se genera el gráfico \emph{Plotly y se guarda correctamente.}}\\\hline
  Excepciones                        & \multicolumn{2}{p{10.25cm}}{Hay un fallo en la carga de la imagen.}\\
  }
  
  \clearpage
  
 \tablaSmallSinColores{C2: Obtener información de los defectos}{p{3cm} p{.75cm} p{9.5cm}}{casodeUso2}{
  \multicolumn{3}{l}{C2: Obtener información de los defectos.} \\
 }
 {
  Descripción                       &  \multicolumn{2}{p{10.25cm}}{Gracias al gráfico \emph{Plotly} el usuario puede colocar el cursor sobre los diferentes defectos y observar la información correspondiente. También puede pulsar sobre la leyenda para mostrar/ocultar cada uno.} \\\hline
   \multirow{2}{3.5cm}{Requisitos}   &   \multicolumn{2}{p{10.25cm}}{RF 1.3} \\\cline{2-3}
                                    &   \multicolumn{2}{p{10.25cm}}{RF 2}\\\hline
                                    &   \multicolumn{2}{p{10.25cm}}{RF 2.1}\\\hline
                                    &   \multicolumn{2}{p{10.25cm}}{RF 2.2}\\\hline
                                    &   \multicolumn{2}{p{10.25cm}}{RF 2.3}\\\hline
                                    &   \multicolumn{2}{p{10.25cm}}{RF 3.1}\\\hline
                                    &   \multicolumn{2}{p{10.25cm}}{RF 3.2}\\\hline
  Precondiciones                    &   \multicolumn{2}{p{10.25cm}}{Detección ejecutada.}   \\\hline
  \multirow{2}{3.5cm}{Secuencia normal}  & Paso & Acción \\\cline{2-3}
                                         & 1    & Mostrando el gráfico, colocar el cursor sobre cada defecto coloreado.
  \\\cline{2-3}
                                         & 2    & Se muestra el ID del defecto, Área y tipo clasificado por tamaño, ya sea Grande, Pequeño o Mediano.
  \\\cline{2-3}
                                         & 3    & Se puede pulsar y mantener pulsado para hacer zoom sobre la imagen o defectos.
  \\\cline{2-3}
                                         & 4    & Se muestran 3 opciones de descarga de resultados: Gráfico interactivo, máscara binaria ó composición original-máscara.
                                         \\\hline
  Postcondiciones                        & \multicolumn{2}{p{10.25cm}}{El usuario ha podido comprobar la información de cada defecto detectado.}\\\hline
  Excepciones                        & \multicolumn{2}{p{10.25cm}}{No hay detecciones y no se muestran defectos pero si el gráfico \emph{Plotly}}\\
  }
  
\tablaSmallSinColores{C4: Consultar el histórico}{p{3cm} p{.75cm} p{9.5cm}}{casodeUso4}{
  \multicolumn{3}{l}{C4: Consultar el histórico.} \\
 }
 {
  Descripción                       &  \multicolumn{2}{p{10.25cm}}{Permite cargar una imagen a la aplicación para su uso.} \\\hline
  \multirow{2}{3.5cm}{Requisitos}   &   \multicolumn{2}{p{10.25cm}}{RF 4} \\\cline{2-3}
                                    &   \multicolumn{2}{p{10.25cm}}{RF 4.1} \\\cline{2-3}
                                    &   \multicolumn{2}{p{10.25cm}}{RF 4.2}\\\hline
  Precondiciones                    &   \multicolumn{2}{p{10.25cm}}{Aplicación inicializada}   \\\hline
  \multirow{2}{3.5cm}{Secuencia normal}  & Paso & Acción \\\cline{2-3}
                                         & 1    & En la \emph{Navbar} superior, se selecciona la opción Histórico.
  \\\cline{2-3}
                                         & 2    & Se muestra el gráfico \emph{Plotly} y es posible hacer zoom, alejar la vista y colocar el cursor sobre los puntos para obtener la información de cada día. Defectos de un determinado tamaño detectados ese día y número total de imágenes procesadas.
                                         \\\hline
  Postcondiciones                        & \multicolumn{2}{p{10.25cm}}{El usuario comprueba como es el histórico de detecciones en la aplicación.} \\\hline
  Excepciones                        & \multicolumn{2}{p{10.25cm}}{El fichero histórico está ausente o defectuoso.}\\
}