\documentclass[a4paper,12pt,twoside]{memoir}

% Castellano
\usepackage[spanish,es-tabla]{babel}
\selectlanguage{spanish}
\usepackage[utf8]{inputenc}
\usepackage[T1]{fontenc}
\usepackage{lmodern} % scalable font
\usepackage{microtype}
\usepackage{placeins}
\usepackage{eurosym}
\usepackage{dirtree}

\RequirePackage{booktabs}
\RequirePackage[table]{xcolor}
\RequirePackage{xtab}
\RequirePackage{multirow}

% Links
\PassOptionsToPackage{hyphens}{url}\usepackage[colorlinks]{hyperref}
\hypersetup{
	allcolors = {red}
}

% Ecuaciones
\usepackage{amsmath}

% Rutas de fichero / paquete
\newcommand{\ruta}[1]{{\sffamily #1}}

% Párrafos
\nonzeroparskip

% Huérfanas y viudas
\widowpenalty100000
\clubpenalty100000

% Evitar solapes en el header
\nouppercaseheads

% Imagenes
\usepackage{graphicx}
\newcommand{\imagen}[2]{
	\begin{figure}[!h]
		\centering
		\includegraphics[width=0.9\textwidth]{#1}
		\caption{#2}\label{fig:#1}
	\end{figure}
	\FloatBarrier
}

\newcommand{\imagenflotante}[2]{
	\begin{figure}%[!h]
		\centering
		\includegraphics[width=0.9\textwidth]{#1}
		\caption{#2}\label{fig:#1}
	\end{figure}
}



% El comando \figura nos permite insertar figuras comodamente, y utilizando
% siempre el mismo formato. Los parametros son:
% 1 -> Porcentaje del ancho de página que ocupará la figura (de 0 a 1)
% 2 --> Fichero de la imagen
% 3 --> Texto a pie de imagen
% 4 --> Etiqueta (label) para referencias
% 5 --> Opciones que queramos pasarle al \includegraphics
% 6 --> Opciones de posicionamiento a pasarle a \begin{figure}
\newcommand{\figuraConPosicion}[6]{%
  \setlength{\anchoFloat}{#1\textwidth}%
  \addtolength{\anchoFloat}{-4\fboxsep}%
  \setlength{\anchoFigura}{\anchoFloat}%
  \begin{figure}[#6]
    \begin{center}%
      \Ovalbox{%
        \begin{minipage}{\anchoFloat}%
          \begin{center}%
            \includegraphics[width=\anchoFigura,#5]{#2}%
            \caption{#3}%
            \label{#4}%
          \end{center}%
        \end{minipage}
      }%
    \end{center}%
  \end{figure}%
}

%
% Comando para incluir imágenes en formato apaisado (sin marco).
\newcommand{\figuraApaisadaSinMarco}[5]{%
  \begin{figure}%
    \begin{center}%
    \includegraphics[angle=90,height=#1\textheight,#5]{#2}%
    \caption{#3}%
    \label{#4}%
    \end{center}%
  \end{figure}%
}
% Para las tablas
\newcommand{\otoprule}{\midrule [\heavyrulewidth]}
%
% Nuevo comando para tablas pequeñas (menos de una página).
\newcommand{\tablaSmall}[5]{%
 \begin{table}
  \begin{center}
   \rowcolors {2}{gray!35}{}
   \begin{tabular}{#2}
    \toprule
    #4
    \otoprule
    #5
    \bottomrule
   \end{tabular}
   \caption{#1}
   \label{tabla:#3}
  \end{center}
 \end{table}
}

%
%Para el float H de tablaSmallSinColores
\usepackage{float}

%
% Nuevo comando para tablas pequeñas (menos de una página).
\newcommand{\tablaSmallSinColores}[5]{%
 \begin{table}[H]
  \begin{center}
   \begin{tabular}{#2}
    \toprule
    #4
    \otoprule
    #5
    \bottomrule
   \end{tabular}
   \caption{#1}
   \label{tabla:#3}
  \end{center}
 \end{table}
}

\newcommand{\tablaApaisadaSmall}[5]{%
\begin{landscape}
  \begin{table}
   \begin{center}
    \rowcolors {2}{gray!35}{}
    \begin{tabular}{#2}
     \toprule
     #4
     \otoprule
     #5
     \bottomrule
    \end{tabular}
    \caption{#1}
    \label{tabla:#3}
   \end{center}
  \end{table}
\end{landscape}
}

%
% Nuevo comando para tablas grandes con cabecera y filas alternas coloreadas en gris.
\newcommand{\tabla}[6]{%
  \begin{center}
    \tablefirsthead{
      \toprule
      #5
      \otoprule
    }
    \tablehead{
      \multicolumn{#3}{l}{\small\sl continúa desde la página anterior}\\
      \toprule
      #5
      \otoprule
    }
    \tabletail{
      \hline
      \multicolumn{#3}{r}{\small\sl continúa en la página siguiente}\\
    }
    \tablelasttail{
      \hline
    }
    \bottomcaption{#1}
    \rowcolors {2}{gray!35}{}
    \begin{xtabular}{#2}
      #6
      \bottomrule
    \end{xtabular}
    \label{tabla:#4}
  \end{center}
}

%
% Nuevo comando para tablas grandes con cabecera.
\newcommand{\tablaSinColores}[6]{%
  \begin{center}
    \tablefirsthead{
      \toprule
      #5
      \otoprule
    }
    \tablehead{
      \multicolumn{#3}{l}{\small\sl continúa desde la página anterior}\\
      \toprule
      #5
      \otoprule
    }
    \tabletail{
      \hline
      \multicolumn{#3}{r}{\small\sl continúa en la página siguiente}\\
    }
    \tablelasttail{
      \hline
    }
    \bottomcaption{#1}
    \begin{xtabular}{#2}
      #6
      \bottomrule
    \end{xtabular}
    \label{tabla:#4}
  \end{center}
}

%
% Nuevo comando para tablas grandes sin cabecera.
\newcommand{\tablaSinCabecera}[5]{%
  \begin{center}
    \tablefirsthead{
      \toprule
    }
    \tablehead{
      \multicolumn{#3}{l}{\small\sl continúa desde la página anterior}\\
      \hline
    }
    \tabletail{
      \hline
      \multicolumn{#3}{r}{\small\sl continúa en la página siguiente}\\
    }
    \tablelasttail{
      \hline
    }
    \bottomcaption{#1}
  \begin{xtabular}{#2}
    #5
   \bottomrule
  \end{xtabular}
  \label{tabla:#4}
  \end{center}
}



\definecolor{cgoLight}{HTML}{EEEEEE}
\definecolor{cgoExtralight}{HTML}{FFFFFF}

%
% Nuevo comando para tablas grandes sin cabecera.
\newcommand{\tablaSinCabeceraConBandas}[5]{%
  \begin{center}
    \tablefirsthead{
      \toprule
    }
    \tablehead{
      \multicolumn{#3}{l}{\small\sl continúa desde la página anterior}\\
      \hline
    }
    \tabletail{
      \hline
      \multicolumn{#3}{r}{\small\sl continúa en la página siguiente}\\
    }
    \tablelasttail{
      \hline
    }
    \bottomcaption{#1}
    \rowcolors[]{1}{cgoExtralight}{cgoLight}

  \begin{xtabular}{#2}
    #5
   \bottomrule
  \end{xtabular}
  \label{tabla:#4}
  \end{center}
}




\graphicspath{ {./img/} }

% Capítulos
\chapterstyle{bianchi}
\newcommand{\capitulo}[2]{
	\setcounter{chapter}{#1}
	\setcounter{section}{0}
	\setcounter{figure}{0}
	\setcounter{table}{0}
	\chapter*{#2}
	\addcontentsline{toc}{chapter}{#2}
	\markboth{#2}{#2}
}

% Apéndices
\renewcommand{\appendixname}{Apéndice}
\renewcommand*\cftappendixname{\appendixname}

\newcommand{\apendice}[1]{
	%\renewcommand{\thechapter}{A}
	\chapter{#1}
}

\renewcommand*\cftappendixname{\appendixname\ }

% Formato de portada
\makeatletter
\usepackage{xcolor}
\newcommand{\tutor}[1]{\def\@tutor{#1}}
\newcommand{\course}[1]{\def\@course{#1}}
\definecolor{cpardoBox}{HTML}{E6E6FF}
\def\maketitle{
  \null
  \thispagestyle{empty}
  % Cabecera ----------------
\noindent\includegraphics[width=\textwidth]{cabecera}\vspace{1cm}%
  \vfill
  % Título proyecto y escudo informática ----------------
  \colorbox{cpardoBox}{%
    \begin{minipage}{.8\textwidth}
      \vspace{.5cm}\Large
      \begin{center}
      \textbf{TFG del Grado en Ingeniería Informática}\vspace{.6cm}\\
      \textbf{\LARGE\@title{}}
      \end{center}
      \vspace{.2cm}
    \end{minipage}

  }%
  \hfill\begin{minipage}{.20\textwidth}
    \includegraphics[width=\textwidth]{escudoInfor}
  \end{minipage}
  \vfill
  % Datos de alumno, curso y tutores ------------------
  \begin{center}%
  {%
    \noindent\LARGE
    Presentado por \@author{}\\ 
    en Universidad de Burgos --- \@date{}\\
    Tutor: \@tutor{}\\
  }%
  \end{center}%
  \null
  \cleardoublepage
  }
\makeatother


% Datos de portada
\title{Detección de defectos en piezas metálicas usando radiografías y \emph{Deep Learning} - \\Documentación Técnica}
\author{Fco. Javier Yagüe Izquierdo}
\tutor{José Francisco Diez Pastor y Pedro Latorre Carmona}
\date{\today}

\begin{document}

\maketitle



\cleardoublepage



%%%%%%%%%%%%%%%%%%%%%%%%%%%%%%%%%%%%%%%%%%%%%%%%%%%%%%%%%%%%%%%%%%%%%%%%%%%%%%%%%%%%%%%%



\frontmatter


\clearpage

% Indices
\tableofcontents

\clearpage

\listoffigures

\clearpage

\listoftables

\clearpage

\mainmatter

\appendix

\apendice{Plan de Proyecto Software}

\section{Introducción}

En este apartado se recoge la planificación temporal que se ha seguido durante el desarrollo del proyecto, analizando los pasos seguidos en cada una de las fases y el estudio de la viabilidad tanto económica como legal que tendría el desarrollo del proyecto. 

\section{Planificación temporal}

En este proyecto se ha aplicado la metodología \emph{Scrum}, por lo que se han ido estableciendo unos objetivos conforme se avanzaba el desarrollo y se han ido dividiendo en \emph{sprints}. Estos normalmente compuestos de \emph{issues} o puntos relevantes a abordar hasta el siguiente \emph{sprint}.

En determinadas ocasiones y en caso de que las tareas llevasen más o menos tiempo, era posible adelantar o retrasar las reuniones semanales para optimizar el tiempo. El objetivo no era hacer muchas tareas simultáneamente sino tener siempre ciertas tareas en proceso para que el desarrollo fuese constante.

Para la gestión del proyecto se utilizó la plataforma de desarrollo colaborativo \emph{Github}. El repositorio puede encontrase en \url{https://github.com/fyi0000/TFG-GII-20.04}. 

Paralelamente y aunque se hizo de forma personal y se ha ido modificando, también se llevó un diario de tareas posibles o conceptos en la plataforma \emph{Trello}. Esta plataforma permite la organización temporal según se personalice en diferentes tarjetas o \emph{boxes} que pueden eliminarse, editarse o asignarse a otras tareas según se considere. 

\subsection{Sprint 0: Introducción y revisión}
Durante este primer \emph{sprint} se concertó una reunión y se estableció un plan de fechas en \emph{Microsoft Teams} para las posteriores semanas. A su vez se introdujo el trabajo anterior de la compañera \emph{Noelia Ubierna Fernández} realizado el año pasado.

Se comentaron las mejoras y principales diferencias respecto al anterior. Marcando como objetivo que el funcionamiento se basase en un conjunto de imágenes propias en lugar de un repositorio de terceros \cite{repositorio:ferguson}.

Por ello primero se establecieron los puntos:

\begin{itemize}
    \item Inicializar el repositorio y la toma de documentación
    \item Descarga de imágenes propias
    \item Descarga y ejecución con las imágenes objetivo actuales en la herramienta del año pasado
\end{itemize}

\subsection{Sprint 1: Conclusiones y decisiones}
Tras las pruebas iniciales en las imágenes que se marcaron como objetivo se observó que el comportamiento no era el más adecuado, si bien esto se esperaba al estar la red neuronal entrenada sobre un conjunto de imágenes distinto al de las imágenes segmentadas por los propios tutores.

Se contemplaron alternativas y modelos parecidos a \emph{Mask R-CNN} que utilizaba la antigua herramienta.

La principal tarea de este \emph{sprint} se estableció en estudiar la viabilidad de utilizar la herramienta \emph{Detectron2}. 

\subsection{Sprint 2: Registro de imágenes y pruebas}
Finalmente se decidió que el proyecto comenzaría desde 0 utilizando \emph{Detectron2}.

El primer paso para su uso era el estudio del formato \emph{COCO} (\emph{Common Objects in Context}) para el registro del conjunto y posterior entrenamiento. Este formato se ha detallado en la memoria del proyecto, consiste en la generación de un fichero \emph{JSON} que recoge las imágenes y sus correspondientes nombres de fichero asignándoles un identificador, clases presentes en el conjunto y las anotaciones. Este último punto especifica como es la forma de un defecto y que espacio ocupa dentro de la imagen además de relacionar cada uno con las distintas imágenes que forman el conjunto.

Por ello los tutores facilitaron el acceso a la computadora de la Universidad \emph{Alpha}.

Los objetivos fueron:

\begin{itemize}
    \item Introducirse y documentarse al registro de imágenes en formato \emph{COCO}
    \item Primeras pruebas con las imágenes del repositorio de \emph{Max Ferguson} \cite{repositorio:ferguson}
\end{itemize}

\subsection{Sprint 3: Cambio de entorno y primeros resultados}
Durante el desarrollo del anterior \emph{sprint} se ``descubrió'' el entorno \emph{Google Colaboratory} y la facilidad de uso del mismo. Si bien ya se conocía la herramienta \emph{Azure} de \emph{Microsoft}. 

La documentación oficial de \emph{Detectron2} y numerosos usuarios utilizaban este entorno así que en lugar de realizar el desarrollo y pruebas en la máquina \emph{Alpha} de la Universidad, se realizaría finalmente en \emph{notebooks} alojados en la plataforma de \emph{Google}. Esto facilitaría el uso repetido ya que no es necesario el uso de \emph{VPN} para su acceso.

Los primeros resultados del registro fueron positivos y se consiguió un \emph{notebook} que:

\begin{itemize}
    \item Recorría las imágenes y estandarizaba los nombres
    \item Generaba el fichero \emph{JSON} con cada una de las secciones
    \item Recorría los directorios y emparejaba las máscaras individualizadas de cada defecto con la imagen correspondiente
\end{itemize}

\subsection{Sprint 4: Primer Entrenamiento}
Durante este \emph{sprint}:

\begin{itemize}
    \item Se completó el registro de todas las imágenes propias
    \item Se comprobó la correcta integración del conjunto con \emph{Detectron2}
    \item Se realizó el primer entrenamiento provisional
\end{itemize}

Los resultados fueron positivos y se planificó continuar las pruebas de entrenamiento, variando valores y distribución de los conjuntos de entrenamiento y test.

\subsection{Sprint 5: Entrenamientos y estudio de gráficas}

\begin{itemize}
    \item Se ajustaron los valores de entrenamiento para mejorar los resultados
    \item Con la herramienta integrada de \emph{Detectron2} llamada \emph{TensorBoard} se observaron las gráficas de \emph{loss} durante entrenamiento y test. Este parámetro representa la eficiencia del modelo sobre el conjunto de entrenamiento y test respectivamente además de servir como referencia para determinar fenómenos como el sobreentrenamiento.
    \item Se observaron las primeras anomalías de entrenamiento
\end{itemize}

En este paso ya se comenzaron a comprobar efectos de sobreentrenamiento en el conjunto.

También se implementaron las métricas \emph{Precision, Recall} y F1 para evaluar el rendimiento del modelo sobre las máscaras reales.

\subsection{Sprint 6: Refinamiento y despliegue web}
Una vez obtenido un modelo eficiente y que cumpliese los mínimos establecidos, se decidió que la ejecución de la herramienta se realizaría mediante el \emph{microframework} \emph{Flask} que utiliza el lenguaje \emph{Python}.

Es en este punto cuando se observa el desarrollo en \emph{sprints} marcados como hitos en el repositorio \cite{repositorio:propio}

A partir de este punto se empiezan a registrar los avances en el repositorio, ya que se hizo uno de prueba y la multitud de cambios realizados en simplemente dos \emph{notebooks} no representaban demasiados avances.

A su vez, para estudiar el comportamiento del modelo dependiendo de la forma del registro de los modelos, 1 imagen por defecto ó 1 imagen conteniendo todos los defectos, se elaboró otro \emph{notebook} basado en el anterior que dividía nuestras máscaras binarias. 
El proceso consistía en reconocer las secciones conexas de la imagen, que presentaban defectos y generar una imagen propia conteniendo dicha región. Además los nombres se generaban de forma que fuese sencillo de relacionar cada imagen original con las múltiples asociadas. No significó diferencia en los resultados.

\subsection{Sprint 7: Avances en \emph{Flask}}
Los sucesivos avances marcados como los dos primeros \emph{milestones} en el repositorio consistieron en:

\begin{itemize}
    \item Generar una sencilla web en \emph{Flask} que a su vez pudiese contenerse en una imagen \emph{Docker}
    \item Probar el funcionamiento de los \emph{Dockerfile}
    \item Añadir gráficos \emph{Plotly} tanto para la presentación de resultados como para el histórico
    \item Añadir \emph{slider} que permite establecer el mínimo de \emph{score} o confianza para que se muestre un defecto
    \item Estudiar el funcionamiento de \emph{Bootstrap}
\end{itemize}

Se generó también una clase propia que manipulase \emph{Detectron2} y permitiese su interacción con la aplicación web.

\subsection{Sprint 8: Finalización y últimos ajustes}
En este \emph{sprint} se finalizó la aplicación web y se ajustó \emph{Bootstrap} para la mejora de la apariencia general de la misma.
Se comenzaron a depurar errores que surgieron en la aplicación meramente estéticos con la introducción de \emph{Bootstrap}

Se realizaron comprobaciones finales y se comenzó con la documentación al tener un contenido consistente.

\clearpage

\section{Estudio de viabilidad}
A continuación se estudiará la viabilidad económica del proyecto contemplando los diferentes costes de personal y equipo y la viabilidad legal que engloba el uso de licencias gratuitas o la compra de las mismas.

\subsection{Viabilidad económica}
Esta sección se dividirá por un lado el coste relacionado con las personas implicadas y por otro, el equipo junto con software y hardware utilizado.

\subsubsection{Coste de personal}
Para el desarrollo del proyecto se han observado dos fases relevantes, 
\begin{itemize}
    \item Entrenamiento y \emph{Deep Learning}
    \item Desarrollo Web
\end{itemize}
Sin embargo, a pesar de ser dos fases casi independientes, la dependencia la una de la otra es tal que es perfectamente viable el desarrollo completo por parte de un único trabajador. No se tiene en cuenta posteriores mejoras como la segmentación de más imágenes para las que podría ser necesario personal más especializado.

\subsubsection{Salario de personal}
Para la consulta del salario de un desarrollador que fuese capaz de hacer ambas tareas se ha consultado la web \emph{PayScale} que analiza perfiles y ofrece estadísticas salariales según puesto, experiencia o rama que se ocupa en un campo.\url{https://www.payscale.com/research/ES/Job=Web_Developer/Salary}

A pesar de recoger distintos perfiles, se tendrá en cuenta la rama general de \emph{Web Developer} en España con 1-2 años de experiencia. 
Esto significa un sueldo al mes de 18000\(/12\)\ =1500\euro.

\tablaSmallSinColores{Costes de personal}{p{6.0cm} p{2.0cm} p{8cm}}{salario}{
  \multicolumn{1}{p{4.5cm}}{\textbf{Concepto}} & \textbf{Valor{}}\\
 }{
  Salario anual bruto  & \multicolumn{1}{r}{18.000}\\
  Cotización a la Seguridad Social & \multicolumn{1}{r}{-1.143}\\
  IRPF 10,89\%  & \multicolumn{1}{r}{-1.888}\\\hline
  \textbf{Sueldo Neto Anual}  & \multicolumn{1}{r}{14.968}\\
  }

Por ello si el proyecto se alargase unos 6 meses el coste total en personal sería de la mitad del anual, 7.484 \euro.

\subsubsection{Coste de equipo}
En cuanto al equipo se debe remarcar que \emph{Flask} es un \emph{miniframework} que no está especialmente diseñado para la respuesta a múltiples peticiones y suele utilizarse en un ámbito más local, como es el presente caso.
Si se quisiese desplegar la aplicación en una escala mayor, se debería contemplar la migración a otros \emph{frameworks} o por ejemplo su alojamiento en \emph{Google Colab} que incluso utilizando la versión Pro, el uso prolongado estaría limitado a 24h.

Un servicio de \emph{hosting} limitado y no demasiado tráfico disponible, podría rondar al año en torno a los 100\euro, que no se contabilizarán en el total ya que en un uso industrial, el despliegue a nivel local puede ser suficiente según necesidades.

\clearpage

\subsubsection{Hardware}
Inicialmente y se se hubiese utilizado la máquina \emph{Alpha} el precio a contemplar sería mucho más alto.
Pero como finalmente se ha hecho uso de \emph{Google Colaboratory} y que la versión Pro, de pago, no parece a priori necesaria, se sumarán 500 euros al total. Esto teniendo en cuenta un equipo sencillo que permita un desarrollo web mínimo y acceso a la plataforma de \emph{Google.} Luego:

\tablaSmallSinColores{Costes Finales}{p{6.0cm} p{2.0cm} p{8cm}}{costefinal}{
  \multicolumn{1}{p{4.5cm}}{\textbf{Concepto}} & \textbf{Valor{}}\\
 }{
  Salario de personal  & \multicolumn{1}{r}{7.484}\\
  Equipo mínimo  & \multicolumn{1}{r}{500}\\\hline
  \textbf{Coste Total}  & \multicolumn{1}{r}{7.984}\\
  }
  
\subsection{Viabilidad legal}
En cuanto a la viabilidad legal, tanto las bibliotecas de \emph{Python} como el uso de \emph{Google Colab} permiten el uso gratuito y no presentan problemas para el uso comercial.

Un punto importante es el uso de imágenes, normalmente privadas, que tendría que facilitar bajo permiso una empresa interesada para su uso en el proyecto. Sin dichas imágenes solo se podría recurrir a imágenes de dominio público.

Esto podría representar un serio problema ya que se requiere de una cantidad mínima de imágenes con las características necesarias para que el modelo esté balanceado. Además es común que este tipo de imágenes correspondan a diferentes piezas y harían falta diferentes enfoques de la misma para que el entrenamiento fuese fiable.

Es uno de los puntos más importantes ya que depende de él la eficiencia del producto final.

\clearpage

\subsubsection{Bibliotecas utilizadas}
\begin{table}[h]
	\begin{center}
		\begin{tabular}{>{\centering\arraybackslash}m{5cm} >{\centering\arraybackslash}m{5cm} p{9cm}}
			\textbf{Biblioteca} & \textbf{Licencia}\\ \hline \hline
			OpenCV & MIT License (MIT)\\ \hline
			Requests & Apache Software License (Apache 2.0)\\ \hline
			Numpy & BSD License (BSD)\\ \hline
			Pandas & BSD License (BSD\\ \hline
			Plotly & MIT License (MIT)\\ \hline
			Matplotlib & BSD License (BSD) Compatible\\ \hline
			PIL & Historical Permission Notice and Disclaimer (HPND) (HPND)\\ \hline
			scikit-image & BSD License (Modified BSD)\\ \hline
			Werkzeug & BSD License (Modified BSD)\\ \hline
			gdown & MIT License (MIT)\\ \hline
			urllib & MIT License (MIT)\\ \hline
			wget & Public Domain (Public Domain)\\ \hline
		\end{tabular}
		\caption{Bibliotecas de \emph{Python} utilizadas y sus licencias}
		\label{Licencias}
	\end{center}
\end{table}

Y por último, tanto \emph{Detectron2} como \emph{Docker}, en su versión \emph{Community} tienen licencia \emph{Apache License 2.0}, que permite el uso comercial con los debidos créditos. Repositorio oficial de \emph{Detectron2}: \url{https://github.com/facebookresearch/detectron2}
\apendice{Especificación de Requisitos}

\section{Introducción}
A continuación se exponen los objetivos generales del proyecto y del producto final así como los diferentes requisitos y casos de uso.

\section{Objetivos generales}

\begin{itemize}
    \item Obtención de un modelo que se comporte correctamente con las imágenes propias
    \item Integración en una aplicación web que permita la correcta ejecución del modelo de forma sencilla para el usuario
    \item Mostrar los resultados de una forma amigable y permitir la conservación de los mismos
\end{itemize}

\section{Catalogo de requisitos}
Se detallan ahora los requisitos funcionales de la aplicación.

\begin{itemize}
    \item \textbf{RF 1} Permitir la ejecución del modelo sobre una radiografía mostrando las detecciones.
    \begin{itemize}
	    \item \textbf{RF 1.1} Poder subir la imagen objetivo del usuario a la aplicación.
	    \item \textbf{RF 1.2} Mostrar previsualización de la imagen.
	    \item \textbf{RF 1.3} Muestra de los resultados de forma interactiva.
	    \item \textbf{RF 1.4} Personalizar el valor límite o confianza de los defectos.
	\end{itemize}

	\item \textbf{RF 2} Permitir la descarga de resultados.
    \begin{itemize}
	    \item \textbf{RF 2.1} Descargar la máscara binaria generada.
	    \item \textbf{RF 2.2} Descargar el resultado interactivo de la figura \emph{Plotly}.
	    \item \textbf{RF 2.3} Descargar los resultados en forma de composición.
	\end{itemize}
	
	\item \textbf{RF 3} Mostrar información de las detecciones.
    \begin{itemize}
	    \item \textbf{RF 3.1} Clasificar los defectos por tamaño y etiquetarlos en la figura.
	    \item \textbf{RF 3.2} Mostrar el área del defecto.
	\end{itemize}
	
	\item \textbf{RF 4} Generar un histórico de detecciones.
    \begin{itemize}
	    \item \textbf{RF 4.1} Registrar en un fichero las detecciones por fecha y tamaño de defectos.
	    \item \textbf{RF 4.2} Generar un gráfico de áreas a partir del histórico.
	\end{itemize}

\end{itemize}

\clearpage

\section{Especificación de requisitos}

\subsection{Diagrama de casos de uso}

\begin{figure}[htb]
	\centering
	\includegraphics[width=1.0\textwidth]{casosdeUso}
	\caption[Diagrama de casos de uso]{Diagrama de casos de uso}
\end{figure}

\tablaSmallSinColores{C0: Cargar imagen a la aplicación}{p{3cm} p{.75cm} p{9.5cm}}{casodeUso0}{
  \multicolumn{3}{l}{C0: Cargar imagen a la aplicación.} \\
 }
 {
  Descripción                       &  \multicolumn{2}{p{10.25cm}}{Permite cargar una imagen a la aplicación para su uso.} \\\hline
  \multirow{2}{3.5cm}{Requisitos}   &   \multicolumn{2}{p{10.25cm}}{RF 1.1} \\\cline{2-3}
                                    &   \multicolumn{2}{p{10.25cm}}{RF 1.2}\\\hline
  Precondiciones                    &   \multicolumn{2}{p{10.25cm}}{Aplicación inicializada}   \\\hline
  \multirow{2}{3.5cm}{Secuencia normal}  & Paso & Acción \\\cline{2-3}
                                         & 1    & Se selecciona una imagen.
  \\\cline{2-3}
                                         & 2    & Se pulsa el botón subir y se muestra la profesionalización.
  \\\cline{2-3}
                                         & 3    & La imagen se guarda en el directorio de la aplicación.
                                         \\\hline
  Postcondiciones                        & \multicolumn{2}{p{10.25cm}}{La imagen se guarda en el directorio correspondiente y se muestra la previsualización.} \\\hline
  Excepciones                        & \multicolumn{2}{p{10.25cm}}{No se ha seleccionado ningún fichero. El formato de fichero no es ni .png, .jpg o .jpeg}\\
}

\clearpage
  
\tablaSmallSinColores{C1: Ejecutar detección}{p{3cm} p{.75cm} p{9.5cm}}{casodeUso1}{
  \multicolumn{3}{l}{C1: Ejecutar detección.} \\
 }
 {
  Descripción                       &  \multicolumn{2}{p{10.25cm}}{Se ejecuta el modelo sobre la imagen y se muestran los resultados.} \\\hline
  \multirow{2}{3.5cm}{Requisitos}   &   \multicolumn{2}{p{10.25cm}}{RF 1} \\\cline{2-3}
                                    &   \multicolumn{2}{p{10.25cm}}{RF 1.4}\\\hline
                                    
  Precondiciones                    &   \multicolumn{2}{p{10.25cm}}{Imagen correctamente cargada en la aplicación.}   \\\hline
  \multirow{2}{3.5cm}{Secuencia normal}  & Paso & Acción \\\cline{2-3}
                                         & 1    & Se selecciona en el \emph{slider} la confianza deseada para la detección.
  \\\cline{2-3}
                                         & 2    & Se pulsa el botón Aceptar y se procede a la detección.
\\\cline{2-3}
                                         & 3    & Tras la carga, se muestran los resultados en forma de gráfico \emph{Plotly}.
                                         \\\hline
  Postcondiciones                        & \multicolumn{2}{p{10.25cm}}{Se genera el gráfico \emph{Plotly y se guarda correctamente.}}\\\hline
  Excepciones                        & \multicolumn{2}{p{10.25cm}}{Hay un fallo en la carga de la imagen.}\\
  }
  
  \clearpage
  
 \tablaSmallSinColores{C2: Obtener información de los defectos}{p{3cm} p{.75cm} p{9.5cm}}{casodeUso2}{
  \multicolumn{3}{l}{C2: Obtener información de los defectos.} \\
 }
 {
  Descripción                       &  \multicolumn{2}{p{10.25cm}}{Gracias al gráfico \emph{Plotly} el usuario puede colocar el cursor sobre los diferentes defectos y observar la información correspondiente. También puede pulsar sobre la leyenda para mostrar/ocultar cada uno.} \\\hline
   \multirow{2}{3.5cm}{Requisitos}   &   \multicolumn{2}{p{10.25cm}}{RF 1.3} \\\cline{2-3}
                                    &   \multicolumn{2}{p{10.25cm}}{RF 2}\\\hline
                                    &   \multicolumn{2}{p{10.25cm}}{RF 2.1}\\\hline
                                    &   \multicolumn{2}{p{10.25cm}}{RF 2.2}\\\hline
                                    &   \multicolumn{2}{p{10.25cm}}{RF 2.3}\\\hline
                                    &   \multicolumn{2}{p{10.25cm}}{RF 3.1}\\\hline
                                    &   \multicolumn{2}{p{10.25cm}}{RF 3.2}\\\hline
  Precondiciones                    &   \multicolumn{2}{p{10.25cm}}{Detección ejecutada.}   \\\hline
  \multirow{2}{3.5cm}{Secuencia normal}  & Paso & Acción \\\cline{2-3}
                                         & 1    & Mostrando el gráfico, colocar el cursor sobre cada defecto coloreado.
  \\\cline{2-3}
                                         & 2    & Se muestra el ID del defecto, Área y tipo clasificado por tamaño, ya sea Grande, Pequeño o Mediano.
  \\\cline{2-3}
                                         & 3    & Se puede pulsar y mantener pulsado para hacer zoom sobre la imagen o defectos.
  \\\cline{2-3}
                                         & 4    & Se muestran 3 opciones de descarga de resultados: Gráfico interactivo, máscara binaria ó composición original-máscara.
                                         \\\hline
  Postcondiciones                        & \multicolumn{2}{p{10.25cm}}{El usuario ha podido comprobar la información de cada defecto detectado.}\\\hline
  Excepciones                        & \multicolumn{2}{p{10.25cm}}{No hay detecciones y no se muestran defectos pero si el gráfico \emph{Plotly}}\\
  }
  
\tablaSmallSinColores{C4: Consultar el histórico}{p{3cm} p{.75cm} p{9.5cm}}{casodeUso4}{
  \multicolumn{3}{l}{C4: Consultar el histórico.} \\
 }
 {
  Descripción                       &  \multicolumn{2}{p{10.25cm}}{Permite cargar una imagen a la aplicación para su uso.} \\\hline
  \multirow{2}{3.5cm}{Requisitos}   &   \multicolumn{2}{p{10.25cm}}{RF 4} \\\cline{2-3}
                                    &   \multicolumn{2}{p{10.25cm}}{RF 4.1} \\\cline{2-3}
                                    &   \multicolumn{2}{p{10.25cm}}{RF 4.2}\\\hline
  Precondiciones                    &   \multicolumn{2}{p{10.25cm}}{Aplicación inicializada}   \\\hline
  \multirow{2}{3.5cm}{Secuencia normal}  & Paso & Acción \\\cline{2-3}
                                         & 1    & En la \emph{Navbar} superior, se selecciona la opción Histórico.
  \\\cline{2-3}
                                         & 2    & Se muestra el gráfico \emph{Plotly} y es posible hacer zoom, alejar la vista y colocar el cursor sobre los puntos para obtener la información de cada día. Defectos de un determinado tamaño detectados ese día y número total de imágenes procesadas.
                                         \\\hline
  Postcondiciones                        & \multicolumn{2}{p{10.25cm}}{El usuario comprueba como es el histórico de detecciones en la aplicación.} \\\hline
  Excepciones                        & \multicolumn{2}{p{10.25cm}}{El fichero histórico está ausente o defectuoso.}\\
}
\apendice{Especificación de diseño}

\section{Introducción}
En esta sección se expone cómo se manipulan los datos en la aplicación, diseño procedimental y el diseño arquitectónico.

\section{Diseño de datos}
A continuación se detalla cómo se organizan los datos en la aplicación. Las imágenes que se han usado para el entrenamiento y test, cómo está estructurada la aplicación con sus ficheros y por último cómo se estructura el fichero que proporciona cierta persistencia entre las ejecuciones.

\subsection{Imágenes} 
Las imágenes utilizadas han sido segmentadas por los tutores de este proyecto. Son un total de 21 imágenes facilitadas en el repositorio. 
Contienen diferentes piezas metálicas que presentan determinados defectos, a su vez se acompañan de 21 máscaras binarias correspondientes a cada una de las anteriores y que presentan una imagen de mismas dimensiones que la original. El contenido es una imagen en 2 colores (binaria) correspondiendo el negro a la pieza y las regiones blancas a los defectos etiquetados.

Asimismo se facilitan unas imágenes con ningún o apenas defectos para la comprobación del comportamiento en estos casos.

\subsection{Aplicación y clases} 
La aplicación utiliza los ficheros y repositorio de \emph{Detectron2}, por lo que en el \emph{container} de \emph{Docker} deberá estar en el mismo directorio para su carga.

El proyecto consta de por un lado 2 ficheros \emph{Python}, 5 ficheros \emph{HTML} y un último fichero de estilo \emph{CSS}.

\subsubsection{Ficheros \emph{Python}}
En esta sección no se entrará en detalle en el funcionamiento interno de \emph{Detectron2}, pero se ha de destacar que se hace uso de la clase \emph{DefaultPredictor} para instanciar el objeto que devolverá los resultados de la imagen.

\subsubsection{Clase Detector}

Clase muy sencilla que se inicializa con la configuración por defecto para ejecutar la detección. Para ello recibe el fichero pesos del modelo a ejecutar y un parámetro confianza que marcará el mínimo de \emph{score} o puntuación que requiere un defecto para ser considerado como tal en los resultados. 
Generada la configuración, instancia un \emph{DefaulPrecidtor} de \emph{Detectron2} con la configuración indicada en el método \emph{inference()}, carga la imagen, ejecuta la detección de la imagen y devuelve los resultados. 

Los métodos \emph{getOutputMask()} y \emph{getConfidence()} son auxiliares y devuelven por un lado la máscara binaria generada de la detección y el vector de \emph{scores} o confianza de cada defecto detectado en caso de detectarse alguno. Por defecto el umbral es de un 70\%.

\begin{figure}[htb]
	\centering
	\includegraphics[width=0.9\textwidth]{claseDetector}
	\caption[Clase Detector]{Clase Detector}
\end{figure}

\subsubsection{Fichero app.py}
Este fichero contiene la aplicación en \emph{Flask} como tal y es la que conecta el objeto \emph{Detector} de la clase anterior con el usuario. Gestiona los ficheros \emph{HTML} y la visualización de la aplicación web. Está dividido en \emph{@app.routes} que indican en qué sección de la aplicación se ejecuta cada método y si responde a peticiones \emph{POST}, por ejemplo.

Es una clase muy extensa y a pesar de estar relativamente modularizada con los métodos, se derivó más de los mismos a esta misma desde la clase \emph{Detector} al presentar algunos problemas como la generación de la propia máscara completa en la clase que podría no detectar el mismo orden al cargarse en la aplicación.

El objeto \emph{Detector} se instancia en cada ejecución. Esto se planteó como un problema ya que gracias a por ejemplo el patrón \emph{Singleton} se podría reutilizar el objeto ya instanciado y optimizar el uso. La principal razón de la estructura actual es que si se quiere cargar la configuración del usuario en el \emph{DefaultPredicor} debe de ser al instanciar de nuevo el objeto. 

\subsubsection{Ficheros \emph{HTML}}
Son la parte \emph{Frontend} de la aplicación y contienen el correspondiente código para la generación de cada una de las partes de la web.
Se ha optado por añadir \emph{Bootsrap} para una mejor apariencia y unas secciones simples de \emph{JavaScript} con peticiones \emph{Ajax} que hacen más interactiva la web en lugar de tener que recargar cada fichero \emph{HTML} cada vez que el usuario realiza una acción.

\subsubsection{Fichero \emph{CSS}}
Contiene el estilo de los ficheros \emph{HTML} para controlar aspectos como la posición y apariencia de los mismos.

\subsection{Fichero registro} 
Fichero \emph{.csv} simple que almacena un identificador de fila, número de imágenes procesadas en un mismo día y 3 columnas que contabilizan cuantos defectos de cada tamaño se han detectado. Se va actualizando conforme avanza la ejecución.

\section{Diseño procedimental}

A continuación se muestra el diagrama de secuencia que detalla la iteración del usuario con la aplicación y cada una de sus secciones.

\begin{figure}[htb]
	\centering
	\includegraphics[width=1.1\textwidth]{diagramaSecuencia}
	\caption[Diagrama de secuencia de la aplicación]{Diagrama de secuencia de la aplicación}
\end{figure}

\clearpage

\section{Diseño arquitectónico}

La estructura de la aplicación se divide en la parte de \emph{Backend} ó código \emph{Python} y \emph{Frontend} que es el contenido de la web y sus ficheros.

\subsection{Backend}
Los ficheros de la aplicación deben de estar inmediatamente situados en el directorio donde se sitúa la carpeta \emph{detectron2} de la cual se cargará la herramienta. Tanto el fichero \emph{app.py} como \emph{detector.py} deben de estar en el mismo directorio. El fichero \emph{registro.csv} también deberá estar en el mismo directorio o la aplicación creará uno de nuevo.

\subsection{Frontend}
Junto al fichero \emph{app.py} que representa la aplicación \emph{Flask} se encuentran los siguientes directorios:

\begin{itemize}
    \item static
        \begin{itemize}
            \item uploads: Directorio donde se almacenarán las imágenes resultado una vez ejecutado el modelo.
            \item css: Contiene el fichero \emph{style.css} de estilo web 
        \end{itemize}
    \item templates: Contiene los ficheros \emph{HTML} que representan cada sección de la aplicación
\end{itemize}

Esta estructura se detallará en el siguiente apartado, mostrando la estructura de directorios.
\apendice{Documentación técnica de programación}

\section{Introducción}
En este apartado se expone la estructura del proyecto, manual para el programador, una guía de instalación y correcta ejecución del proyecto, además de pruebas realizadas sobre su funcionamiento.

\section{Estructura de directorios}
La estructura se contempla una vez ejecutado el fichero \emph{Dockerfile} que construye la imagen que contiene, por un lado el entorno original facilitado por \emph{Detectron2} en su \emph{Dockerfile} oficial que se ha modificado para este proyecto y además el repositorio de \emph{Github} descargado y extraído.


\begin{figure}[h]
	\dirtree{%
		.1 /.
		.2 detectron2.
		.2 configs.
		.2 imagenes
		.2 setup.py.
		.2 proyecto.
		.3 src.
		.3 ....
		.2 descargaModelo.py.
		.2 requirements.txt.
		.2 modelos.json.
		.2 app.py.
		.2 descargaModelo.py.
		.2 detector.py.
		.2 registro.csv.
		.2 templates.
		.3 base.html.
		.3 faq.html.
		.3 deteccion.html.
		.3 index.html.
		.3 historico.html.
		.2 static.
		.3 css.
		.4 style.css.
		.3 uploads.
		.4 graficoHistorico.html.
		.2 modelo-0.1.pth.
	}
	\caption{Estructura del Proyecto}
	\label{esquemadirectorios}
\end{figure}

La carpeta \textbf{proyecto} contiene el proyecto directamente clonado, pero el fichero \emph{Dockerfile} dispone todo según se corresponde automáticamente, no obstante el contenido de \emph{proyecto/src} está tanto en dicho directorio como en el directorio padre.

Además es importante remarcar la presencia de los siguientes ficheros:

\begin{itemize}
    \item \textbf{app.py} y \textbf{detector.py} deberán estar a la misma altura de directorios que la carpeta \textbf{detectron2}
    \item Si no está presente \textbf{registro.csv} en dicho directorio se creará uno nuevo perdiendo la información guardada hasta ese momento
    \item La descarga del modelo es bastante pesada, y aunque cabía la posibilidad de lanzar el fichero \textbf{descargaModelo.py} si no se encuentra el modelo para así asegurar la ejecución, el proceso de descarga se puede demorar por lo que se obvia la presencia del fichero \emph{modelo-0.1.pth}
\clearpage
    \item Para poder mostrar un histórico y en caso de que no haya valores suficientes, el fichero \emph{graficoHistórico.html} debe estar presente desde el primer momento
\end{itemize}

\section{Manual del programador}
En este apartado se detallará cómo está organizado el proyecto y qué ficheros se han utilizado en el desarrollo para que en un futuro pueda ser comprendido correctamente y ayude en el proceso de edición o mejora.

\subsection{Ficheros y directorios}

\begin{itemize}
    \item \textbf{src/app.py:} Fichero principal de la aplicación que contiene el código \emph{Flask}, está estructurado en \emph{routes} que identifican cada método o función con una sección de la web y las peticiones a las que responde.
    A nivel de configuración y más allá del contenido de los propios métodos, se puede personalizar la dirección IP en la que se servirá la aplicación web y el modo \emph{debug} en la parte inferior de la aplicación. 
    
    \item \textbf{src/detector.py:} Fichero que contiene la clase del detector creado para este proyecto. Utiliza el objeto \emph{DefaultPredictor} de \emph{Detectron2} para instanciarlo, darle una configuración o \emph{cfg} y ejecutar la detección sobre el fichero que recibe.
    
    \item \textbf{src/static/css/style.css:} Fichero de estilos para los archivos \emph{HTML} en \emph{templates}. Si se modifica la ruta deberá de modificarse también la referencia en dichos ficheros.
    
    \item \textbf{src/static/uploads/graficoHistórico.html:} Fichero generado por \emph{Plotly} y que por si mismo contiene todo el código para ser interactivo. Puede abrirse en el navegador de forma independiente.
    
    \item \textbf{src/templates/base.html:} Fichero \emph{HTML} que se utiliza principalmente para organizar la \emph{navbar} de \emph{Bootsrap} y las secciones que comparten todas las páginas, de esa forma el resto de ficheros \emph{HTML} extienden de este mismo y puede modificarse a la vez todos las secciones comunes desde este fichero.
    
    \item \textbf{Dockerfile:} Fichero para la importación y ejecución creado a partir del fichero oficial de \emph{Detectron2} disponible en su repositorio\footnote{https://github.com/facebookresearch/detectron2}. 
    Partiendo de una imágen de \emph{NVIDIA} se establece un usuario, un directorio padre y se clona los ficheros de \emph{Detectron2} instalando las dependencias. 
    
    Posteriormente desde \emph{Github} se importa este proyecto y se estructuran los ficheros más importantes. Por último con el fichero \emph{descargaModelo.py} se descarga desde \emph{Google Drive} el fichero pesos debido a que \emph{Github} no permite alojarlo debido a su tamaño. Una vez finalizado el proceso la imagen está lista para generar contenedores.
    
    Se ha probado a generar imágenes de menor tamaño usando por ejemplo la imagen oficial de \emph{Docker Hub} de \emph{Debian} pero puede mostrar incompatibilidades con \emph{Detectron2} y por ello no se ha utilizado finalmente.
    
    \item \textbf{registro.csv:} Fichero que contiene el registro de detecciones por tamaño y fecha.
    \item \textbf{modelo.pth:} Fichero de pesos generado por la red neuronal.
    \item \textbf{detectron2/:} Directorio oficial de \emph{Detectron2}
    \item \textbf{Conversor y Registro COCO.ipynb:} \emph{Notebook} creado para el recorrido, procesado y generación del fichero \emph{JSON} para el registro del conjunto de imágenes en el proceso de entrenamiento y test.
    \item \textbf{Separador de Mascaras.ipynb:} \emph{Notebook} que se creó con la idea de observar la diferencia de utilizar máscaras únicas o individuales. Partiendo de un fichero de máscara binaria, genera tantos como defectos contenga de forma individual.
    \item \textbf{requirements.txt:} Fichero de texto que contiene las bibliotecas requeridas para este proyecto, el fichero \emph{Dockerfile} se encarga de recorrerlo e instalarlo automáticamente en el despliegue para que el usuario no tenga que hacerlo.
    \item \textbf{modelos.json:} Fichero de diccionarios que contiene los modelos registrados con la estructura : \emph{Nombre:URL} de forma que la aplicación al iniciarse descarga el fichero y lo sobrescribe, lo recorre y comprueba el número de versión para descargar la última versión. Dichos modelos también son alojados en \emph{Google Drive} debido a su tamaño.
    \item \textbf{updateDocker.bat:} Fichero informal que se creó a la hora de trabajar con \emph{Pycharm} y \emph{Visual Studio Code}. Ya que en Windows no es posible la ejecución y pruebas del proyecto, contiene un par de líneas que permiten copiar el contenido completo del proyecto al contenedor \emph{Docker} ejecutándose y actualizando el contenido para su prueba inmediata.
    Los comandos siguen la estructura: 
    
   \emph{docker cp directorioHostOrigen nombreContenedor:directorioDestinoContenedor}
    
\end{itemize}


\section{Compilación, instalación y ejecución del proyecto}
En esta sección se describirá el proceso para la instalación y uso del proyecto de cara al desarrollo y modificación por parte de un programador. Para la instalación y uso por parte del usuario hay más detalles en la sección \nameref{instalaciondeusuario}

Hay 2 alternativas para la instalación del proyecto con intención de modificarlo:

\begin{itemize}
    \item Instalar mediante el \emph{Dockerfile} la imagen que se facilita y que contendrá el proyecto, generar un contenedor y mantenerlo en ejecución mientras se hacen los cambios normalmente desde el \emph{host} con un editor más amigable, ya que la interfaz de los contenedores es algo tosca.
    \item Inicializar un directorio en una cuenta personal en \emph{Google Colab} que finalmente se almacenará en la unidad de \emph{Google Drive} de la cuenta asociada
\end{itemize}

Debido a que el primer punto se cubrirá de forma más detallada en la sección de usuario, se procederá ahora a explicar cómo se podría preparar un directorio personal en \emph{Google Drive} y clonar el proyecto para editar y ejecutarlo desde \emph{Google Colab.}

Sin embargo este método tiene un problema y al alojar nuestros ficheros en \emph{Google Drive}, no podemos hacer uso de un editor de texto o \emph{IDE} que nos facilite la edición de código. Una solución puede ser la sincronización de \emph{Visual Studio Code} con \emph{Google Drive}, pero ya que en este proyecto se ha trabajado con el proyecto local, no se abordará esta situación.

\begin{enumerate}
    \item \textbf{Creación del directorio:} Primero se accede mediante una cuenta personal \emph{Gmail} a la unidad de \emph{Google Drive} donde según se considere se creará un directorio nuevo para una mejor organización.
    
    \item \textbf{\emph{Google Colab} y creación del \emph{Notebooks}:} Con este método se trabajará de forma similar a la de \emph{Jupyter} de \emph{Anaconda}, que es posible que el lector esté familiarizado.
    
    Se accede a \emph{Google Colab}\footnote{https://colab.research.google.com/notebooks/intro.ipynb} y seguramente se presente un \emph{Notebook} de introducción. Se crea uno nuevo con un nombre a elegir libremente pero manteniendo la extensión \emph{.ipynb}.
    
    \item \textbf{Disposición e instalación:} El primer paso es integrar la unidad de \emph{Google Drive} con el \emph{Notebook}.
    
    \begin{figure}[htb]
	\centering
	\includegraphics[width=0.6\textwidth]{p1}
	\caption[Integración de Google Drive]{Integración de Google Drive}
    \end{figure}
    
    Las diferentes celdas se ejecutan con un icono a la izquierda aunque desde el menú \emph{Entorno de ejecución} es posible ejecutarlas todas.
    
    Es posible que el código de la figura superior nos pregunte por un código generado por seguridad en un enlace, teniendo iniciada la sesión de \emph{Google} basta con visitar la dirección y copiar y pegar el código que se facilita. Hecho esto se montará la unidad de \emph{Google Drive}.
    
    A partir de aquí es posible recorrer los directorios con \emph{cd} ó \emph{ls} para ver el contenido.
    
    \begin{figure}[htb]
	\centering
	\includegraphics[width=0.9\textwidth]{p2}
	\caption[Recorrido de directorios]{Recorrido de directorios}
    \end{figure}
    
    A continuación instalamos e importamos \emph{pyyaml} y \emph{torch}
    
    \begin{figure}[htb]
	\centering
	\includegraphics[width=0.9\textwidth]{p3}
	\caption[Instalación de dependencias]{Instalación de dependencias}
    \end{figure}
    
    Posteriormente se clona e instala el repositorio oficial de \emph{Detectron2}
    
    \begin{figure}[htb]
	\centering
	\includegraphics[width=0.9\textwidth]{p4}
	\caption[Instalación de Detectron2]{Instalación de Detectron2}
    \end{figure}
    
    Este proceso se puede demorar y hay que tener en cuenta de que al no estar en un entorno virtual, puede haber dependencias no instaladas, por lo que es posible que al ejecutar determinadas celdas se nos solicite la instalación de alguna biblioteca mediante \emph{pip install}. Dichos comandos pueden ejecutarse normalmente de forma individual por celda pero si se ejecutan varios deben de ir precedidos por el símbolo \emph{!} de cierre de exclamación.
    
    Por último se importan bibliotecas y clases básicas de \emph{Detectron2} como se recomienda en la documentación oficial.
    
    \begin{figure}[htb]
	\centering
	\includegraphics[width=0.9\textwidth]{p5}
	\caption[Importación de bibliotecas básicas]{Importación de bibliotecas básicas}
    \end{figure}
    
    \item \textbf{Clonación del proyecto:} Posteriormente se clona el proyecto con un comando de \emph{git} asegurándonos de estar dentro del directorio de \emph{Detectron2}
    
    \emph{! git clone https://github.com/fyi0000/TFG-GII-20.04}
    
    \item \textbf{Organización del proyecto:} Una vez clonado el proyecto, se deben de disponer los ficheros de forma que \emph{app.py, detector.py, descargarModelo.py, static} y \emph{templates} estén a la misma altura que el directorio \emph{detectron2}.
    
    Si hay algún problema de importación simplemente usamos \emph{cd} ó \emph{ls} para asegurarnos que la disposición es correcta.
    
    \item \textbf{Uso de ngrok:} \emph{ngrok} nos permite ejecutar de forma local en \emph{Colab} una aplicación \emph{Flask} y a la vez acceder como si se estuviese ejecutando en nuestra máquina, para ello primero instalamos la biblioteca con:
    
    \emph{!pip install flask-ngrok}
    
    Y posteriormente editamos el fichero \emph{app.py} añadiendo estas líneas en la cabecera
    
    \begin{figure}[htb]
	\centering
	\includegraphics[width=0.6\textwidth]{p6}
	\caption[Importación de bibliotecas básicas]{Importación de bibliotecas básicas}
    \end{figure}
    
    Ahora cuando se ejecute la aplicación, además de la traza normal de \emph{Flask} nos facilitará una \emph{URL} temporal acabada en \emph{ngrok.io} que nos llevará a la aplicación.
    
    \begin{figure}[htb]
	\centering
	\includegraphics[width=0.3\textwidth]{p7}
	\caption[Importación de bibliotecas básicas]{Importación de bibliotecas básicas}
    \end{figure}
    
    Conviene recordar además que la ejecución continuada en \emph{Google Colab} está limitada y podría dejar de ejecutarse nuestra aplicación si se mantiene en ejecución durante demasiado tiempo. El límite es de unos 60 minutos.
    
\end{enumerate}

\clearpage

\subsection{Funcionamiento concurrente de la aplicación}
A nivel de programador es necesario tener presente que \emph{Flask} solo admite por defecto una petición de forma síncrona, por lo que como se ha observado en las pruebas que se comentan en la siguiente sección, las detecciones concurrentes no son una opción por defecto.

Sin embargo y a juicio del lector, es posible indicar en el fichero \emph{app.py} a \emph{Flask} que se admitan diferentes peticiones simultáneas, aunque se avisa de que el rendimiento se puede degradar notablemente.

La forma más común es añadir en la última fila del fichero \emph{app.py}, en el método \emph{\_\_init\_\_} donde tiene lugar el lanzamiento y se indica el puerto, la frase \emph{threaded=True} de forma que el resultado sería:

\begin{center}\emph{app.run(host=0.0.0.0, threaded=True)}\end{center}

\clearpage

\section{Pruebas del sistema}

\begin{table}[h]
	\begin{center}
		\begin{tabular}{>{\centering\arraybackslash}m{5cm} >{\centering\arraybackslash}m{5cm} p{9cm}}
		    \hline
			\textbf{Prueba Realizada}  & \textbf{Resultado}\\ \hline \hline
    			    Iniciar la aplicación mediante ``sudo python app.py'' & La aplicación funciona con normalidad.\\ \hline
    			    Iniciar la aplicación mediante ``python app.py'' & La aplicación ha funcionado en ocasiones pero al no tener permisos de escritura lanza errores. \\ \hline
    			    Acceder a la sección de \emph{FAQ}(Preguntas frecuentes) & Se accede correctamente y los hiperenlaces son accesibles. \\ \hline
    			    Detección de una imagen \emph{test} & La aplicación funciona y muestra resultados. Las descargas funcionan correctamente. \\ \hline
    			    Detección de una imagen \emph{training} & La aplicación funciona con normalidad y los resultados son excesivamente buenos, como se esperaba. \\ \hline
    			    Detección en una radiografía sin defectos & Se muestran los resultados y se indica correctamente que no hay detecciones. Algunas marcadas por los tutores aparentemente sin defectos parecen tener alguno pequeño. \\ \hline
    			    Detección con imagen ajena al conjunto & La aplicación funciona y a veces detecta falsos positivos y a veces no detecta nada y lo indica correctamente. \\ \hline
    			   
		\end{tabular}
		\caption{Pruebas de la aplicación}
		\label{tablapruebas1}
	\end{center}
\end{table}

\begin{table}[h]
	\begin{center}
		\begin{tabular}{>{\centering\arraybackslash}m{5cm} >{\centering\arraybackslash}m{5cm} p{9cm}}
		    \hline
			\textbf{Prueba Realizada}  & \textbf{Resultado}\\ \hline \hline
   		            Detección de imagen con extensión \emph{PNG} & La detección es correcta. \\ \hline
			    	Detección de imagen con extensión \emph{JPG} & La detección es correcta. \\ \hline
    			    Detección de imagen con extensión \emph{JPEG} & La detección es correcta. \\ \hline
			    	Detección de fichero \emph{PDF} & La aplicación rechaza el fichero por extensión correctamente. \\ \hline
			    	Acceso desde dos pestañas concurrentes & La aplicación funciona correctamente. \\ \hline
			    	Detección de dos ficheros de forma concurrente & Como es de esperar, \emph{Flask} solo admite una petición y muestra en ambas el resultado de la última imagen que le ha sido enviada. \\ \hline
    			    Descarga de resultados & Se descargan correctamente, algo más de demora en la composición pero todo funciona según requisitos. \\ \hline
        			Borrado del fichero ``modelo-0.1.pth'' y detección & Se muestra un error web y en la consola se muestra la traza de la excepción y remarcado el mensaje de captura preguntando si existe dicho fichero. \\ \hline
        			Actualización del registro tras detección & Se actualiza la fila correspondiente o se añade una fecha nueva de forma correcta. \\ \hline
		\end{tabular}
		\caption{Pruebas de la aplicación}
		\label{tablapruebas2}
	\end{center}
\end{table}


\begin{table}[h]
	\begin{center}
		\begin{tabular}{>{\centering\arraybackslash}m{5cm} >{\centering\arraybackslash}m{5cm} p{9cm}}
		    \hline
			\textbf{Prueba Realizada}  & \textbf{Resultado}\\ \hline \hline
			    	Actualización del modelo & La aplicación descarga el nuevo fichero y actualiza la variable sesión que indica que fichero se usará en la próxima instanciación del Detector. \\ \hline
        			Actualización del modelo con enlace erróneo & Al comprobarse de forma independiente tras el proceso la existencia del fichero mostrado en \emph{modelos.json}, al no encontrarse se indica que no se ha podido actualizar. \\ \hline
        			Reinicio tras actualización & Se muestra la versión actual correcta y se añade que es la más actual. \\ \hline
        			Detección de una imagen de dimensiones reducidas & Al ser tan pequeño, \emph{Plotly} genera un gráfico de unas dimensiones también muy pequeñas, pero de esta forma se ajusta mejor a imágenes de un tamaño normal. \\ \hline
        			Ejecución de la aplicación sin conexión a Internet & Los CDN no funcionan y la apariencia se degrada, la función de actualización y \emph{loaders} pierden funcionalidad. \\ \hline

    			    
		\end{tabular}
		\caption{Pruebas de la aplicación}
		\label{tablapruebas2}
	\end{center}
\end{table}
\apendice{Documentación de usuario}

\section{Introducción}
A continuación se explica cómo es el proceso de instalación completo del proyecto, desde la instalación de \emph{Docker} hasta el uso del fichero \emph{Dockerfile} que se recuerda está en el repositorio del proyecto\footnote{Repositorio: https://github.com/fyi0000/TFG-GII-20.04}.

\section{Requisitos de usuarios}
Es imprescindible que, al menos para la instalación y actualización del modelo se disponga de \textbf{conexión a Internet.}

Además se debe contemplar que tanto \emph{jQuery} como \emph{Bootsrap} se utilizan mediante un \emph{CDN} (\emph{Content Delivery Network}), que evita la descarga de los ficheros para que ambos componentes funcionen. Por ello de no haber conexión a internet determinadas peticiones no podrían resolverse y la aplicación perdería funcionalidades.

Como ya se ha expuesto anteriormente, oficialmente \emph{Detectron2} no es compatible con Windows, por lo que, y según su web oficial\cite{detecron2:instalacion} los requisitos son:

\begin{itemize}
    \item \emph{Linux} o \emph{MacOs} con una versión de \emph{Python} 3.6 o superior
    \item \emph{Pytorch} y \emph{torchvision} versión 1.6 o superior
    \item \emph{OpenCV} es opcional pero se recomienda para la visualización
\end{itemize}

Además, por la forma como está elaborado el proyecto, se requiere tener instalado \emph{Docker}. La instalación que se detalla no es para nada compleja pero es para \emph{Windows} y puede variar en el entorno Linux o similares aunque los comandos de uso son los mismos.


Una vez generada la imagen ocupa un total de unos 7.5-8GB de espacio en disco. Se han hecho pruebas con la imagen sobre la que se genera la imagen, siendo esta de \emph{Nvidia} y la indicada por \emph{Detectron2}. A pesar de probar con imágenes como \emph{Debian}, instalar las dependencias no es suficiente y puede generar errores. Por ello no ha sido posible minimizar más el espacio.

\section{Instalación} \label{instalaciondeusuario}

\begin{enumerate}
    \item El primer paso es acceder a la web de \emph{Docker} y en concreto a la sección de \emph{Windows}: \\
    
    \begin{center}\url{https://docs.docker.com/docker-for-windows/install/}\end{center}
    
    La instalación es prácticamente automática y solo hay que esperar a que se complete.
    
    \item Una vez instalado, se aconseja utilizar \emph{WSL 2 based engine} que es una alternativa a \emph{Hyper V} para la virtualización.
    
    \textbf{Nota importante:} En los equipos que se han utilizado ha sido además necesario la instalación de una actualización del kernel de \emph{Linux} y posterior reinicio. Puede no ser el caso pero si \emph{Docker} no inicia correctamente, se adjunta el instalador de la última versión en el soporte digital y si no acceder a  a: \\
    
    \url{https://docs.docker.com/docker-for-windows/wsl/}\\
    \begin{center}
    ó\end{center}
    \url{https://docs.microsoft.com/es-es/windows/wsl/install-win10#step-4---download-the-linux-kernel-update-package}
    
    \clearpage
    
    Se ha de tener en cuenta que los comandos que se detallan a continuación y en general todas las funciones de \emph{Docker} no funcionarán salvo que se esté ejecutando el \emph{daemon}. Por ello confirmar que no hay una actualización en curso y que está funcionando correctamente si se observa el siguiente icono estático en la barra de tareas
    
    \begin{figure}[htb]
	\centering
	\includegraphics[width=0.1\textwidth]{iconodocker}
	\caption[Icono de ejecución Docker]{Icono de ejecución Docker}
    \end{figure}
    
    \item Una vez instalado y teniendo \emph{Docker} en ejecución, se debería de clonar el proyecto desde el repositorio mediante un comando:\\
    
    \begin{center}\emph{git clone https://github.com/fyi0000/TFG-GII-20.04}\end{center}
    

    O bien desde las unidades digitales facilitadas que contienen los mismos ficheros más el modelo. Al final se tendría que observar un directorio similar al siguiente:
    
    \begin{figure}[htb]
	\centering
	\includegraphics[width=0.5\textwidth]{manual1}
	\caption[Directorio tras clonación o copia]{Directorio tras clonación o copia}
    \end{figure}
    
    \clearpage
        
    \item A partir de este punto se recomienda utilizar el \emph{PowerShell} para llegar al directorio donde se encuentra el fichero \emph{Dockerfile} sin extensión, aunque también es posible utilizar la Consola normal.
    
    Una forma rápida es pulsar \emph{Shift Izquierdo + Botón derecho} y se mostrará esta opción:
    
    \begin{figure}[htb]
	\centering
	\includegraphics[width=0.6\textwidth]{manual2}
	\caption[Opción de apertura directa PowerShell]{Opción de apertura directa PowerShell}
    \end{figure}
    
    Que como se puede comprobar nos permite abrir la ventana en el propio directorio sin necesidad de utilizar \emph{cd's}.
    
    \item\label{paso5} A continuación ejecutaremos el comando \emph{Docker} que a partir del \emph{Dockerfile} facilitado, construye la imagen a partir de la cual se pueden generar los contenedores.
    El comando es: \\
    
    \begin{center}\emph{docker build . -t nombreimagen}\end{center}
    
    El punto a continuación de \emph{build} indica que el fichero está en el directorio actual, aunque como se puede suponer es posible ejecutar la consola en cualquier directorio y \emph{apuntar} al fichero \emph{Dockerfile} pasando su ruta como este primer directorio.
    A continuación de \emph{-t} o \emph{tag} se especifica un nombre de la imagen en minúscula.
    
    \begin{figure}[htb]
	\centering
	\includegraphics[width=0.8\textwidth]{manual3}
	\caption[Comando build de imagen]{Comando build de imagen}
    \end{figure}
    
    \item Debido al tamaño de la imagen y que además se instalan todas las dependencias contenidas en el fichero \emph{requirements.txt}, el proceso puede superar los 10 minutos hasta que se complete la creación de la imagen.
    
    Aunque la traza del proceso simplemente indica lo que está haciendo, es importante por ejemplo, y si es posible, asegurarse de que el comando ``RUN python descargaModelo.py'' se demora un tiempo ya que nos indica si ha podido haber algún problema con el modelo alojado en \emph{Google Drive}.
    
    \begin{figure}[htb]
	\centering
	\includegraphics[width=0.8\textwidth]{manual4}
	\caption[Descarga de modelo]{Descarga de modelo}
    \end{figure}
    
 
    \item Terminado el proceso, abrimos \emph{Docker} y nos vamos a la sección \emph{images} en la parte superior izquierda donde deberíamos ver una imagen de entorno a 8GB y con el nombre o \emph{tag} que le hemos dado en el paso 5 de este apartado.
    
    \begin{figure}[htb]
	\centering
	\includegraphics[width=1.0\textwidth]{manual5}
	\caption[Sección Images en Docker]{Sección Images en Docker}
    \end{figure}
    
    \item Confirmado que la imagen se ha creado correctamente, pulsamos sobre el botón \emph{RUN} que aparece al colocar el cursor sobre la imagen.
    
    \begin{figure}[htb]
	\centering
	\includegraphics[width=1.2\textwidth]{manualRunImage}
	\caption[Creación de contenedor desde imagen]{Creación de contenedor desde imagen}
    \end{figure}
    
    \clearpage
    
    Ahora se nos abrirá una pequeña ventana de creación directa. \textbf{No pulsamos Run} y desplegamos las \emph{Optional Settings}. Ahora asignamos un nombre a nuestra elección al contenedor y en \emph{Local Port} o Puerto Local escribimos \textbf{5000}. Hacemos esto ya que por defecto es el puerto expuesto en \emph{Flask}.
    
    \begin{figure}[htb]
	\centering
	\includegraphics[width=1.0\textwidth]{manual6}
	\caption[Configuración del contenedor]{Configuración del contenedor}
    \end{figure}
    
    \item Ahora nos vamos a la sección \emph{Containers/Apps} de \emph{Docker} y confirmamos que está el contenedor que acabamos de crear ejecutándose.
    
    \begin{figure}[htb]
	\centering
	\includegraphics[width=1.1\textwidth]{manual7}
	\caption[Contenedor correctamente creado]{Contenedor correctamente creado}
    \end{figure}
    
    \clearpage
    
    Para comprobar que todo ha funcionado:
    
    \begin{itemize}
        \item El contenedor se mantiene en ejecución
        \item En texto azul y a la derecha del nombre del contenedor está nuestra imagen inicialmente creada
        \item El puerto que se indica debajo del nombre es efectivamente el 5000
    \end{itemize}
    
    A la derecha del nombre nos aparecen diferentes botones, los que se usarán son:
    
    \begin{itemize}
        \item 2º para lanzar la línea de comandos del contenedor
        \item 3º para iniciar o detener el contenedor
        \item 5º para borrar el contenedor, se debe tener en cuenta que no se puede borrar una imagen si hay contenedores dependientes existentes
    \end{itemize}  
    
    Y por último podemos lanzar la línea de comandos, con el 2º botón anteriormente mencionado e iniciando el contenedor si no lo estaba, y hacer un \emph{ls} para comprobar que todo está correctamente estructurado.
    
    \begin{figure}[htb]
	\centering
	\includegraphics[width=1.1\textwidth]{manual8}
	\caption[Directorio correcto]{Directorio correcto}
    \end{figure}
    
    Ficheros que deben estar presentes:
    
    \begin{itemize}
        \item \textbf{app.py}: Es la aplicación y debe estar presente en este punto.
        \item \textbf{detector.py}
        \item\textbf{descargaModelo.py}: Es el script auxiliar que descarga el modelo inicialmente.
        \item\textbf{detectron2}: Directorio de \emph{Detectron2} necesario para la ejecución de la aplicación. Nos indica que se ha clonado correctamente e instalado.
        \item\textbf{modelo-0.1.pth}: Modelo inicial que debe estar presente siempre.
        \item\textbf{static y templates}: Directorios que contienen la estructura web luego han de estar localizados aquí.
        \item\textbf{registro.csv}: Fichero que inicializa el histórico.
        \item\textbf{static/uploads/graficoHistorico.html}: Histórico inicial
    \end{itemize}  
    
    De no estar estos ficheros o estar en otra disposición puede dar lugar a errores.
    
\end{enumerate}

Como punto final es posible que se quiera limpiar la \emph{caché} del \emph{builder} que aunque \emph{Docker} la limpia, se puede hacer de forma inmediata con: \\
\begin{center}\emph{docker builder prune}\end{center}

    \begin{figure}[htb]
	\centering
	\includegraphics[width=1.1\textwidth]{manual30}
	\caption[Limpieza de la cache de instalación]{Limpieza de la cache de instalación}
    \end{figure}
    
Confirmando que realmente se desea limpiar la caché de toda la instalación de esta sección.

Hasta este punto se considera la instalación de la herramienta, el uso y manual correspondiente se detallará en el apartado que viene a continuación.

\clearpage

\section{Manual del usuario}
En esta sección se detalla el uso como usuario de la aplicación en cada una de sus partes.

\subsubsection{Ejecución y acceso}
Suponiendo que el proceso de la sección anterior haya funcionado correctamente, ya se está en disposición de iniciar la aplicación.

Primero se iniciará la aplicación \emph{Docker} se encontraba ya en ejecución y de igual manera el contenedor anteriormente creado.

Ahora se accede mediante el 2º botón del contenedor a la línea de comandos.
Para iniciar la aplicación y asegurase de que la ejecución tiene los permisos, se hará con la palabra \emph{sudo}, ya que se ha probado a ajustar permisos con \emph{chmod} en el propio \emph{Dockerfile} pero han surgido problemas de igual manera.

\begin{center}\emph{sudo python app.py}\end{center}

\textbf{Nota:} Es posible configurar el contenedor para el lanzamiento inmediato de nuestra aplicación, para ello basta con añadir al \emph{Dockerfile} la línea

\begin{center}\emph{\#CMD [``python'', ``app.py'']}\end{center}

que se encuentra comentada en el \emph{Dockerfile} facilitado. El problema que presenta es que la ejecución no siempre es como \emph{sudo} y de haber un fallo, el contenedor se detendrá y tendremos que reconstruir la imagen de nuevo desde el \emph{Dockerfile}. Por ello se ha optado por dejar la ejecución de forma manual.

    \begin{figure}[htb]
	\centering
	\includegraphics[width=1.1\textwidth]{manual9}
	\caption[Lanzamiento de la aplicación]{Lanzamiento de la aplicación}
    \end{figure}
    
\emph{Flask} nos indica ahora que es un \emph{microframework} y por lo tanto el mensaje que no recomienda su uso en producción es normal, dado que no se recomienda someterlo a múltiples peticiones. Para este proyecto su rendimiento es perfectamente suficiente luego no hay problema.

En la parte inferior nos indica la dirección donde se ejecuta nuestra aplicación web seguido del puerto que se expuso cuando se configuró el contenedor. Accedemos desde el navegador escribiendo la dirección o bien:

\begin{center}\emph{http://localhost:5000/}\end{center}

Una vez accedemos se nos muestra el inicio con una pequeña sección de información

    \begin{figure}[htb]
	\centering
	\includegraphics[width=1.1\textwidth]{manual10}
	\caption[Página de inicio]{Página de inicio}
    \end{figure}

A su vez en la parte superior izquierda tenemos una \emph{navbar} con la que podemos navegar de forma rápida por todas las secciones

    \begin{figure}[htb]
	\centering
	\includegraphics[width=0.8\textwidth]{manual11}
	\caption[Navbar superior]{Navbar superior}
    \end{figure}

\clearpage

\subsubsection{Detección}
Si queremos empezar con la detección, podemos acceder desde el botón \emph{Detectar} que se nos muestra el inicio o desde la correspondiente opción de la \emph{navbar}.

En esta sección se nos muestra un \emph{input} para ficheros que no permite continuar a no ser que se haya facilitado un fichero.

    \begin{figure}[htb]
	\centering
	\includegraphics[width=0.8\textwidth]{manual13}
	\caption[Fichero requerido]{Fichero requerido}
    \end{figure}
    
A su vez las restricciones de imagen debido a los formatos que se ha comprobado que funcionan en \emph{Detectron2} son \emph{PNG, JPG y JPEG}, rechazando cualquier otro fichero.

    \begin{figure}[htb]
	\centering
	\includegraphics[width=0.8\textwidth]{manual14}
	\caption[Formatos requeridos]{Formatos requeridos}
    \end{figure}
    
\clearpage

Si la imagen que se facilita es correcta se obtiene una previsualización de la misma y un \emph{slider} para marcar la confianza o seguridad mínima que debe de tener un defecto detectado para ser mostrado en los resultados.

    \begin{figure}[htb]
	\centering
	\includegraphics[width=0.8\textwidth]{manual15}
	\caption[Previsualización de imagen a detectar]{Previsualización de imagen a detectar}
    \end{figure}

Si todo es correcto se pulsa \emph{Aceptar} y comienza el proceso mostrando un \emph{loader}.

    \begin{figure}[htb]
	\centering
	\includegraphics[width=0.3\textwidth]{manual16}
	\caption[Loader de detección]{Loader de detección}
    \end{figure}

Una vez finaliza el proceso, se muestra un gráfico \emph{Plotly} mostrando los resultados y detecciones.

    \begin{figure}[htb]
	\centering
	\includegraphics[width=1.0\textwidth]{manual17}
	\caption[Muestra de resultados]{Muestra de resultados}
    \end{figure}

Si se coloca el cursor sobre algún defecto, se obtiene además información sobre el mismo como el área que ocupa, tamaño que se le otorga según el área, confianza de la detección y en el borde gris el identificador del defecto.

    \begin{figure}[htb]
	\centering
	\includegraphics[width=0.4\textwidth]{manualHoverInfo}
	\caption[Muestra de métricas]{Muestra de métricas}
    \end{figure}

\clearpage

El gráfico \emph{Plotly} tiene multitud de opciones y al colocar el cursor sobre él apareciendo diferentes controles

    \begin{figure}[htb]
	\centering
	\includegraphics[width=0.6\textwidth]{manual18}
	\caption[Controles Plotly]{Controles Plotly}
    \end{figure}


Aunque son descriptivos por si mismos, algunas opciones son:

\begin{itemize}
    \item Descargar como .png la perspectiva actual
    \item Aumentar el zoom o seleccionar herramienta mover
    \item Manipular la leyenda y restaurar el zoom inicial
    \item Enlace a la web de \emph{Plotly}
\end{itemize}

Además es posible hacer zoom arrastrando y soltando el cursor sobre una zona específica

    \begin{figure}[htb]
	\centering
	\includegraphics[width=0.8\textwidth]{manual19}
	\caption[Función de zoom]{Función de zoom}
    \end{figure}

\clearpage

Pudiendo ahora aplicar otra herramienta interesante, si se hace click sobre la leyenda y en concreto el identificador de un defecto que queremos ocultar temporalmente.

    \begin{figure}[htb]
	\centering
	\includegraphics[width=0.8\textwidth]{manual20}
	\caption[Ocultando detecciones]{Ocultando detecciones}
    \end{figure}

Basta con volverá hacer click en dicha leyenda para que se vuelva a mostrar.

A su vez se muestran opciones de descarga de resultados, Máscara binaria por un lado y por otro tanto el propio gráfico \emph{Plotly} que puede abrirse independientemente en el navegador como una composición imagen original - máscara binaria.

    \begin{figure}[htb]
	\centering
	\includegraphics[width=0.6\textwidth]{manual22}
	\caption[Botones de descarga]{Botones de descarga}
    \end{figure}

\clearpage

La máscara binaria:

    \begin{figure}[htb]
	\centering
	\includegraphics[width=0.6\textwidth]{manual21}
	\caption[Máscara binaria descargada]{Máscara binaria descargada}
    \end{figure}

Composición:

    \begin{figure}[htb]
	\centering
	\includegraphics[width=0.4\textwidth]{manual23}
	\caption[Composición descargada]{Composición descargada}
    \end{figure}
    
\clearpage

En caso de no haber detecciones se permitirá igualmente la observación con \emph{Plotly} y la descarga del gráfico interactivo, indicando al lado del botón que los resultados están vacíos y por lo tanto no se han detectado defectos.

    \begin{figure}[htb]
	\centering
	\includegraphics[width=1.2\textwidth]{manual28}
	\caption[Ejecución sin detecciones]{Ejecución sin detecciones}
    \end{figure}

\clearpage

\subsubsection{Histórico}

Además si se accede a la sección Histórico se mostrará otro gráfico \emph{Plotly} que muestra la sucesión de detecciones por fecha, dividiendo por el tamaño el número de defectos que se han detectado ese día y mostrando además el número de imágenes procesadas ese día para tener mayor perspectiva.

    \begin{figure}[htb]
	\centering
	\includegraphics[width=1.1\textwidth]{manual24}
	\caption[Gráfico de histórico]{Gráfico de histórico}
    \end{figure}

Igualmente si se pasa el cursor por el gráfico además de las mismas opciones del gráfico resultados, se muestran los valores de la fecha sobre la que se sitúa el cursor.

\clearpage

\subsubsection{Actualización de modelo}

Debido a que en un hipotético caso podrían incorporarse más imágenes al conjunto, sería posible reentrenar la red y obtener un mayor rendimiento. Por ello la aplicación comprueba cada vez que se lanza, el fichero \emph{modelos.json} del repositorio donde se podría colocar en forma de diccionario, un nuevo modelo, indicando la versión y enlace de \emph{Google Drive} para la detección y descarga.

Por defecto y para demostrar el funcionamiento, el \emph{Dockerfile} descarga la versión 0.1 estando disponible la 0.2, por lo que se mostrará este mensaje.

    \begin{figure}[htb]
	\centering
	\includegraphics[width=0.8\textwidth]{manual12}
	\caption[Actualización del modelo]{Actualización del modelo}
    \end{figure}

Si se pulsa el botón \emph{Actualizar Modelo} se iniciará la actualización y se mostrará un pequeño \emph{loader}

    \begin{figure}[htb]
	\centering
	\includegraphics[width=0.2\textwidth]{manual25}
	\caption[Loader de actualización]{Loader de actualización}
    \end{figure}

Si todo funciona aparecerá el siguiente mensaje indicando a qué versión se ha actualizado, si no se indicará que ha habido un problema en la actualización.

    \begin{figure}[htb]
	\centering
	\includegraphics[width=0.5\textwidth]{manual26}
	\caption[Actualización correcta]{Actualización correcta}
    \end{figure}

\clearpage

Y en sucesivas ocasiones que se lance la aplicación, se mostrará un mensaje similar asegurando que la comprobación se ha realizado y el modelo local es el último disponible.

    \begin{figure}[htb]
	\centering
	\includegraphics[width=0.5\textwidth]{manual27}
	\caption[Modelo local actualizado]{Modelo local actualizado}
    \end{figure}

Debido a que la presencia de al menos un modelo es vital para que la aplicación cumpla su función, no se elimina el modelo anterior, lo que conviene tener presente a pesar de que no ocupan demasiado espacio por sí mismos.

\subsection{En caso de no poder descargar el modelo inicial}
En los soportes digitales se facilita el modelo inicial además del script para su descarga. Si por algún motivo no se pudiese descargar y hacer funcionar la aplicación, los pasos serían los siguientes:

\begin{enumerate}
    \item Iniciar el contenedor creado como se ha detallado anteriormente
    \item Mediante el comando \emph{pwd} es posible confirmar que la ruta es \\ \begin{center}
    \emph{home/appuser/detectron2\_repo}\end{center} si no es así navegar con \emph{cd} hasta encontrar \emph{app.py}.
    \item A continuación y en el directorio del anfitrión donde se encuentre el fichero \emph{modelo-0.1.pth}, abrir o navegar mediante una consola común o \emph{PowerShell} y ejecutar:
    \begin{center}\emph{docker cp modelo-0.1.pth nombreContenedor:home/appuser/detectron2\_repo}\end{center}
    Donde \emph{nombreContendor} es el nombre que se le haya dado al contenedor creado.
\end{enumerate}

Prestar atención a los espacios en las rutas ya que podría ser necesario el uso de comillas en el argumento del comando


\bibliographystyle{plain}
\bibliography{bibliografiaAnexos}

\end{document}
